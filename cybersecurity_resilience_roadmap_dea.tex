\documentclass[12pt,a4paper]{report}

% Required packages for official document formatting
\usepackage{geometry}
\usepackage{array}
\usepackage{booktabs}
\usepackage{times}
\usepackage{helvet}
\usepackage{courier}
\usepackage{graphicx}
\usepackage{fancyhdr}
\usepackage{lastpage}
\usepackage{xcolor}
\usepackage{enumitem}
\usepackage{tabularx}
\usepackage{longtable}
\usepackage{multirow}
\usepackage{siunitx}
\usepackage{tocloft}
\usepackage{hyperref}

% Set proper margins for official documents
\geometry{
    left=1in,
    right=1in,
    top=1in,
    bottom=1in
}

% Define custom colors for official document
\definecolor{govblue}{RGB}{0,51,102}
\definecolor{govgray}{RGB}{102,102,102}
\definecolor{secred}{RGB}{178,34,34}
\definecolor{secgreen}{RGB}{0,128,0}
\definecolor{secorange}{RGB}{255,140,0}

% Remove paragraph indentation
\setlength{\parindent}{0pt}
\setlength{\parskip}{4pt}

% Header and footer setup
\pagestyle{fancy}
\fancyhf{}
\renewcommand{\headrulewidth}{0.5pt}
\renewcommand{\footrulewidth}{0pt}
\fancyhead[L]{\color{govblue}\textbf{DEA Cybersecurity Resilience Roadmap}}
\fancyhead[R]{\textit{Confidential}}
\rfoot{Page \thepage\ of \pageref{LastPage}}

% Custom commands for official formatting
\newcommand{\sectionheader}[1]{\textbf{\color{govblue}#1}}
\newcommand{\initiative}[1]{\textbf{\color{secgreen}\ding{51} #1}}
\newcommand{\progress}[1]{\textit{#1}}
\newcommand{\roadmap}[1]{\textbf{\color{secred}#1}}
\newcommand{\highlight}[1]{\textcolor{secorange}{\textbf{#1}}}

% Table of contents formatting
\renewcommand{\cfttoctitlefont}{\hfill\Large\bfseries}
\renewcommand{\cftaftertoctitle}{\hfill}

\begin{document}

% Ministry Letterhead
\begin{center}
    {\Large \textbf{GOVERNMENT OF INDIA}}\\
    {\large \textbf{MINISTRY OF FINANCE}}\\
    {\large \textbf{DEPARTMENT OF ECONOMIC AFFAIRS}}\\[0.3cm]
    \includegraphics[width=2cm]{ashoka_emblem.png}\\[0.2cm]
    % Note: Replace ashoka_emblem.png with actual emblem file
\end{center}

\vspace{0.3cm}

% Document Title
\begin{center}
    {\Large \textbf{CYBERSECURITY RESILIENCE ROADMAP}}\\
    {\large Department of Economic Affairs}\\[0.2cm]
    {\large \textit{Initiatives, Progress, and Strategic Framework}}
\end{center}

\vspace{0.3cm}

\begin{flushright}
    \textbf{Note No:} DEA/2025/SEC/001\\
    \textbf{Date:} 27.10.2025\\
    \textbf{Classification:} \textcolor{secred}{CONFIDENTIAL}
\end{flushright}

\vspace{0.5cm}

% Executive Summary
\sectionheader{EXECUTIVE SUMMARY}

The Department of Economic Affairs (DEA) is implementing a comprehensive cybersecurity resilience framework to protect India's financial infrastructure and economic governance systems. This roadmap outlines current initiatives, implementation progress, and strategic direction for achieving robust cyber resilience by 2027.

\textbf{Key Strategic Objectives:}
\begin{itemize}[leftmargin=*, itemsep=3pt]
    \item Protection of critical financial infrastructure and economic data
    \item Implementation of zero-trust architecture across all DEA systems
    \item Enhanced threat intelligence and incident response capabilities
    \item Capacity building and awareness programs for all stakeholders
    \item Compliance with national cybersecurity frameworks and international standards
\end{itemize}

\textbf{Investment Framework:} \textbf{Rs. 485 crore} allocated for cybersecurity initiatives over FY 2025-2027, with \textbf{Rs. 178 crore} earmarked for FY 2025-26.

\tableofcontents
\newpage

% Section 1: Current Cybersecurity Landscape
\chapter{CURRENT CYBERSECURITY LANDSCAPE}

\sectionheader{1.1 Threat Environment Assessment}

The Department of Economic Affairs faces an evolving cyber threat landscape characterized by:

\subsection*{1.1.1 Primary Threat Vectors}
\begin{enumerate}[leftmargin=*, itemsep=4pt]
    \item \highlight{Advanced Persistent Threats (APTs)} targeting financial systems
    \item \highlight{Ransomware attacks} on critical economic infrastructure
    \item \highlight{Supply chain compromises} affecting financial software vendors
    \item \highlight{Insider threats} from authorized users with malicious intent
    \item \highlight{Phishing and social engineering} targeting DEA officials
\end{enumerate}

\subsection*{1.1.2 Current Risk Assessment}
\begin{table}[h]
\centering
\begin{tabular}{@{}lcc@{}}
\toprule
\textbf{Risk Category} & \textbf{Current Level} & \textbf{Target Level} \\
\midrule
Data Protection & Medium & Low \\
Network Security & Medium-High & Low \\
Application Security & Medium & Low \\
Incident Response & Medium & Low \\
Compliance & High & Low \\
\bottomrule
\end{tabular}
\end{table}

\sectionheader{1.2 Regulatory Framework}

DEA cybersecurity operations are governed by:
\begin{itemize}[leftmargin=*, itemsep=3pt]
    \item Information Technology Act, 2000 (as amended 2023)
    \item CERT-In (Indian Computer Emergency Response Team) Directions 2022
    \item National Cyber Security Policy 2013 (under revision)
    \item Cybersecurity Framework for Financial Institutions (RBI Guidelines)
    \item Data Protection Bill 2023 (pending implementation)
\end{itemize}

% Section 2: Current Security Architecture
\chapter{CURRENT SECURITY ARCHITECTURE}

\sectionheader{2.1 Infrastructure Overview}

\subsection*{2.1.1 Network Architecture}
DEA operates a multi-tiered network infrastructure:
\begin{itemize}[leftmargin=*, itemsep=3pt]
    \item \textbf{Core Network:} 285 servers across 4 data centers
    \item \textbf{Protected Network:} 142 critical application servers
    \item \textbf{DMZ Network:} 42 public-facing servers
    \item \textbf{User Network:} 1,850+ employee workstations
\end{itemize}

\subsection*{2.1.2 Security Stack}
\begin{table}[h]
\centering
\begin{tabular}{@{}ll@{}}
\toprule
\textbf{Security Layer} & \textbf{Implementation} \\
\midrule
Firewall Protection & Next-Gen Firewalls (Palo Alto Networks) \\
Intrusion Detection & IDS/IPS systems (Cisco Firepower) \\
Endpoint Protection & EDR solutions (CrowdStrike) \\
Email Security & Advanced Threat Protection (Proofpoint) \\
Data Encryption & AES-256 for data at rest and transit \\
Identity Management & Multi-factor authentication \\
Vulnerability Management & Regular penetration testing \\
\bottomrule
\end{tabular}
\end{table}

\sectionheader{2.2 Current Security Operations}

\subsection*{2.2.1 Security Operations Center (SOC)}
\begin{itemize}[leftmargin=*, itemsep=3pt]
    \item \textbf{24/7 Monitoring:} Dedicated SOC with 42 security analysts
    \item \textbf{SIEM Implementation:} IBM QRadar platform for log analysis
    \item \textbf{Threat Intelligence Integration:} Real-time threat feeds from MISP
    \item \textbf{Incident Response:} Average response time: 2.3 hours
\end{itemize}

\subsection*{2.2.2 Compliance and Auditing}
\begin{itemize}[leftmargin=*, itemsep=3pt]
    \item \textbf{ISO 27001:2013} certified for information security management
    \item \textbf{PCI DSS} compliance for payment system security
    \item \textbf{Regular Audits:} Quarterly internal and annual external audits
    \item \textbf{Risk Assessments:} Bi-annual comprehensive risk evaluations
\end{itemize}

% Section 3: Strategic Initiatives
\chapter{STRATEGIC CYBERSECURITY INITIATIVES}

\sectionheader{3.1 Zero Trust Architecture Implementation}

\initiative{Project ZeroGuard (FY 2025-27)}
\progress{Status: Phase I (30\% Complete)}

\subsection*{Objectives:}
\begin{itemize}[leftmargin=*, itemsep=3pt]
    \item Eliminate traditional network perimeter assumptions
    \item Implement identity-centric access controls
    \item Enable continuous monitoring and verification
    \item Minimize lateral movement for potential attackers
\end{itemize}

\subsection*{Implementation Components:}
\begin{table}[h]
\centering
\begin{tabular}{@{}lcc@{}}
\toprule
\textbf{Component} & \textbf{Timeline} & \textbf{Budget (Rs. crore)} \\
\midrule
Identity Access Management & Q1-Q3 FY 2025-26 & 45 \\
Network Segmentation & Q2 FY 2025-26 & 28 \\
Endpoint Security & Q3-Q4 FY 2025-26 & 37 \\
Data Classification & Q1 FY 2026-27 & 22 \\
Continuous Monitoring & Q2-Q4 FY 2026-27 & 41 \\
\bottomrule
\end{tabular}
\end{table}

\sectionheader{3.2 Cloud Security Transformation}

\initiative{Project CloudSecure (FY 2025-27)}
\progress{Status: Planning Phase (15\% Complete)}

\subsection*{Scope:}
\begin{itemize}[leftmargin=*, itemsep=3pt]
    \item Migration of 67\% of applications to secure cloud infrastructure
    \item Implementation of Cloud Security Posture Management (CSPM)
    \item Development of secure DevSecOps practices
    \item Integration with national cloud security framework
\end{itemize}

\subsection*{Cloud Security Architecture:}
\begin{itemize}[leftmargin=*, itemsep=3pt]
    \item \textbf{Infrastructure as Code (IaC):} Terraform-based secure deployments
    \item \textbf{Container Security:} Kubernetes security policies and scanning
    \item \textbf{API Security:} Gateway-level API protection and monitoring
    \item \textbf{Data Protection:} Cloud-native encryption and key management
\end{itemize}

\sectionheader{3.3 Threat Intelligence Program}

\initiative{Project ThreatWatch (FY 2025-26)}
\progress{Status: Implementation Phase (45\% Complete)}

\subsection*{Components:}
\begin{enumerate}[leftmargin=*, itemsep=4pt]
    \item \textbf{Threat Intelligence Platform:} Integration with global threat feeds
    \item \textbf{Dark Web Monitoring:} Continuous monitoring of cybercriminal forums
    \item \textbf{Vulnerability Intelligence:} Proactive vulnerability scanning and assessment
    \item \textbf{Indicators of Compromise (IoC) Database:} Comprehensive IoC repository
\end{enumerate}

\subsection*{Key Metrics:}
\begin{itemize}[leftmargin=*, itemsep=3pt]
    \item \textbf{Threat Feeds:} 250+ integrated sources
    \item \textbf{Daily IoCs Processed:} 15,000+ indicators
    \item \textbf{False Positive Rate:} <5\%
    \item \textbf{Threat Intelligence Sharing:} Active participation in FIN-CSIRT
\end{itemize}

% Section 4: Implementation Progress
\chapter{IMPLEMENTATION PROGRESS}

\sectionheader{4.1 Achievements to Date}

\subsection*{4.1.1 Financial Year 2024-25 Achievements}
\begin{table}[h]
\centering
\begin{tabular}{@{}lcc@{}}
\toprule
\textbf{Initiative} & \textbf{Planned} & \textbf{Completed} \\
\midrule
Security Infrastructure Upgrade & 100\% & 100\% \\
Endpoint Protection Deployment & 100\% & 100\% \\
SIEM Implementation & 100\% & 100\% \\
Security Awareness Training & 95\% & 98\% \\
Vulnerability Management & 100\% & 100\% \\
Incident Response Framework & 100\% & 90\% \\
\bottomrule
\end{tabular}
\end{table}

\subsection*{4.1.2 Key Performance Indicators}
\begin{itemize}[leftmargin=*, itemsep=3pt]
    \item \textbf{Security Incidents:} Reduced by 67\% (from 142 to 47 incidents annually)
    \item \textbf{Mean Time to Detect (MTTD):} Improved from 72 hours to 4.2 hours
    \item \textbf{Mean Time to Respond (MTTR):} Reduced from 48 hours to 2.3 hours
    \item \textbf{Vulnerability Remediation:} 95\% of critical vulnerabilities patched within 7 days
    \item \textbf{Security Awareness:} 98\% staff completion rate for training programs
\end{itemize}

\sectionheader{4.2 Ongoing Projects}

\subsection*{4.2.1 Current Portfolio}
\begin{table}[h]
\centering
\begin{tabular}{@{}lccc@{}}
\toprule
\textbf{Project} & \textbf{Start Date} & \textbf{Completion} & \textbf{Status} \\
\midrule
Zero Trust Implementation & Apr 2025 & Mar 2027 & 30\% \\
Cloud Security Migration & Jul 2025 & Jun 2027 & 15\% \\
Threat Intelligence Platform & Jan 2025 & Dec 2025 & 45\% \\
Security Operations Center Upgrade & Oct 2025 & Mar 2026 & 20\% \\
Supply Chain Security Framework & Jan 2026 & Dec 2026 & 10\% \\
\bottomrule
\end{tabular}
\end{table}

\subsection*{4.2.2 Budget Utilization}
\begin{itemize}[leftmargin=*, itemsep=3pt]
    \item \textbf{FY 2024-25 Allocation:} Rs. 125 crore
    \item \textbf{Expenditure to Date:} Rs. 118 crore (94\% utilization)
    \item \textbf{FY 2025-26 Allocation:} Rs. 178 crore
    \item \textbf{Q1 Utilization:} Rs. 38 crore (85\% of quarterly target)
\end{itemize}

% Section 5: Roadmap to Resilience
\chapter{ROADMAP TO RESILIENCE}

\sectionheader{5.1 Strategic Timeline}

\subsection*{5.1.1 Short-term Objectives (FY 2025-26)}
\roadmap{Foundation Building}
\begin{itemize}[leftmargin=*, itemsep=3pt]
    \item Complete Zero Trust Architecture design and pilot implementation
    \item Establish comprehensive threat intelligence capabilities
    \item Upgrade Security Operations Center with AI-driven analytics
    \item Implement advanced endpoint detection and response
    \item Launch cybersecurity awareness and training program
\end{itemize}

\subsection*{5.1.2 Medium-term Objectives (FY 2026-27)}
\roadmap{Operational Excellence}
\begin{itemize}[leftmargin=*, itemsep=3pt]
    \item Full deployment of Zero Trust Architecture across all systems
    \item Complete cloud security transformation
    \item Implement supply chain security framework
    \item Establish cyber range for training and simulation
    \item Achieve ISO 27001:2022 certification
\end{itemize}

\subsection*{5.1.3 Long-term Objectives (FY 2027-28)}
\roadmap{Strategic Resilience}
\begin{itemize}[leftmargin=*, itemsep=3pt]
    \item Implement predictive threat analytics using AI/ML
    \item Establish quantum-resistant cryptography framework
    \item Develop autonomous security operations capabilities
    \item Achieve full compliance with emerging data protection regulations
    \item Establish DEA as a cybersecurity excellence center
\end{itemize}

\sectionheader{5.2 Investment Framework}

\subsection*{5.2.1 Budget Allocation}
\begin{table}[h]
\centering
\begin{tabular}{@{}lcc@{}}
\toprule
\textbf{Category} & \textbf{FY 2025-26} & \textbf{FY 2026-27} \\
\midrule
Infrastructure Modernization & Rs. 72 crore & Rs. 85 crore \\
Security Operations & Rs. 48 crore & Rs. 62 crore \\
Training and Awareness & Rs. 18 crore & Rs. 22 crore \\
Research and Innovation & Rs. 25 crore & Rs. 38 crore \\
Compliance and Audit & Rs. 15 crore & Rs. 18 crore \\
\textbf{Total} & \textbf{Rs. 178 crore} & \textbf{Rs. 225 crore} \\
\bottomrule
\end{tabular}
\end{table}

\subsection*{5.2.2 Return on Investment}
\begin{itemize}[leftmargin=*, itemsep=3pt]
    \item \textbf{Risk Reduction:} 85\% reduction in cyber risk exposure
    \item \textbf{Operational Efficiency:} 40\% improvement in security operations
    \item \textbf{Cost Avoidance:} Rs. 285 crore projected savings over 5 years
    \item \textbf{Reputation Protection:} Maintaining public trust in financial systems
\end{itemize}

% Section 6: Partnerships and Collaboration
\chapter{PARTNERSHIPS AND COLLABORATION}

\sectionheader{6.1 Government Partnerships}

\subsection*{6.1.1 National Security Ecosystem}
\begin{itemize}[leftmargin=*, itemsep=3pt]
    \item \textbf{CERT-In:} Active participation in national incident response
    \item \textbf{NIC:} Collaboration on secure government network infrastructure
    \item \textbf{NTRO:} Intelligence sharing on emerging cyber threats
    \item \textbf{CBI:} Support for cybercrime investigation and prosecution
\end{itemize}

\subsection*{6.1.2 Financial Sector Collaboration}
\begin{itemize}[leftmargin=*, itemsep=3pt]
    \item \textbf{RBI:} Alignment with financial sector cybersecurity guidelines
    \item \textbf{SEBI:} Coordination on market infrastructure protection
    \item \textbf{IRDAI:} Collaboration on insurance sector security standards
    \item \textbf{PFRDA:} Pension system security framework development
\end{itemize}

\sectionheader{6.2 International Cooperation}

\subsection*{6.2.1 Multilateral Engagement}
\begin{itemize}[leftmargin=*, itemsep=3pt]
    \item \textbf{Financial Action Task Force (FATF):} Cybersecurity standards development
    \item \textbf{World Bank:} Technical assistance and knowledge sharing
    \item \textbf{IMF:} Financial sector cybersecurity best practices
    \item \textbf{ASEAN:} Regional cybersecurity cooperation initiatives
\end{itemize}

\subsection*{6.2.2 Public-Private Partnerships}
\begin{itemize}[leftmargin=*, itemsep=3pt]
    \item \textbf{Technology Partners:} Collaboration with leading security vendors
    \item \textbf{Academic Institutions:} Research partnerships with IITs and IISc
    \item \textbf{Industry Associations:} Knowledge sharing with NASSCOM and DSCI
    \item \textbf{Start-up Ecosystem:} Innovation partnerships with cyber security start-ups
\end{itemize}

% Section 7: Risk Management
\chapter{RISK MANAGEMENT}

\sectionheader{7.1 Risk Assessment Framework}

\subsection*{7.1.1 Risk Categories}
\begin{table}[h]
\centering
\begin{tabular}{@{}lcc@{}}
\toprule
\textbf{Risk Category} & \textbf{Impact} & \textbf{Likelihood} \\
\midrule
Data Breach & Critical & Medium \\
Ransomware Attack & High & Low-Medium \\
Insider Threat & High & Low \\
Supply Chain Compromise & High & Medium \\
Service Disruption & High & Medium \\
Compliance Violation & Medium & Low \\
\bottomrule
\end{tabular}
\end{table}

\subsection*{7.1.2 Risk Mitigation Strategies}
\begin{enumerate}[leftmargin=*, itemsep=4pt]
    \item \textbf{Preventive Controls:} Multi-layered security architecture
    \item \textbf{Detective Controls:} Advanced monitoring and analytics
    \item \textbf{Corrective Controls:} Incident response and recovery procedures
    \item \textbf{Deterrent Controls:} Security awareness and training programs
\end{enumerate}

\sectionheader{7.2 Business Continuity Planning}

\subsection*{7.2.1 Disaster Recovery Capabilities}
\begin{itemize}[leftmargin=*, itemsep=3pt]
    \item \textbf{Recovery Time Objective (RTO):} 4 hours for critical systems
    \item \textbf{Recovery Point Objective (RPO):} 15 minutes maximum data loss
    \item \textbf{Backup Strategy:} 3-2-1 backup rule implementation
    \item \textbf{Failover Sites:} Active-passive configuration with automatic failover
\end{itemize}

\subsection*{7.2.2 Crisis Management Framework}
\begin{itemize}[leftmargin=*, itemsep=3pt]
    \item \textbf{Incident Response Team:} Multi-disciplinary response team
    \item \textbf{Communication Protocols:} Internal and external communication plans
    \item \textbf{Decision Making Framework:} Clear escalation procedures
    \item \textbf{Stakeholder Management:} Coordination with all relevant stakeholders
\end{itemize}

% Section 8: Conclusion
\chapter{CONCLUSION}

\sectionheader{8.1 Strategic Outlook}

The Department of Economic Affairs is committed to building a world-class cybersecurity capability that protects India's financial infrastructure and supports economic growth. The roadmap outlined in this document provides a comprehensive framework for achieving cyber resilience through strategic investments, capability building, and collaborative partnerships.

\sectionheader{8.2 Key Success Factors}

\begin{itemize}[leftmargin=*, itemsep=3pt]
    \item \textbf{Leadership Commitment:} Strong support from senior leadership
    \item \textbf{Resource Allocation:} Adequate funding and human resources
    \item \textbf{Technical Excellence:} Adoption of best-in-class security technologies
    \item \textbf{Continuous Improvement:} Regular assessment and enhancement
    \item \textbf{Collaborative Approach:} Strong partnerships with stakeholders
\end{itemize}

\sectionheader{8.3 Next Steps}

\begin{enumerate}[leftmargin=*, itemsep=4pt]
    \item Obtain approval for the proposed investment framework
    \item Establish project governance structures
    \item Initiate procurement for identified security solutions
    \item Launch comprehensive stakeholder engagement program
    \item Establish monitoring and reporting mechanisms
\end{enumerate}

\vspace{1cm}

\begin{center}
    \textbf{Prepared by:}\\
    \textit{Cybersecurity Division}\\
    \textit{Department of Economic Affairs}\\
    \textit{Ministry of Finance}\\[0.5cm]
    \textbf{Date:} 27.10.2025
\end{center}

\end{document}