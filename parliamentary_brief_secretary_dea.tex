\documentclass[12pt,a4paper]{article}

% Required packages for official document formatting
\usepackage{geometry}
\usepackage{array}
\usepackage{booktabs}
\usepackage{times}
\usepackage{helvet}
\usepackage{courier}
\usepackage{graphicx}
\usepackage{fancyhdr}
\usepackage{lastpage}
\usepackage{xcolor}
\usepackage{enumitem}
\usepackage{tabularx}
\usepackage{longtable}

% Set proper margins for official documents
\geometry{
    left=1in,
    right=1in,
    top=1in,
    bottom=1in
}

% Define custom colors for official document
\definecolor{govblue}{RGB}{0,51,102}
\definecolor{govgray}{RGB}{102,102,102}
\definecolor{restrictedred}{RGB}{178,34,34}

% Remove paragraph indentation
\setlength{\parindent}{0pt}
\setlength{\parskip}{4pt}

% Header and footer setup
\pagestyle{fancy}
\fancyhf{}
\renewcommand{\headrulewidth}{0pt}
\renewcommand{\footrulewidth}{0pt}
\rfoot{Page \thepage\ of \pageref{LastPage}}

% Custom commands for official formatting
\newcommand{\briefheader}[1]{\textbf{\color{restrictedred}#1}}
\newcommand{\sectionheader}[1]{\textbf{#1}}
\newcommand{\datapoint}[1]{\textit{#1}}

\begin{document}

% RESTRICTED Classification Header
\begin{center}
    {\Large \color{restrictedred} \textbf{RESTRICTED - FOR PARLIAMENTARY USE ONLY}}
\end{center}

\vspace{0.5cm}

% Brief Header
\begin{center}
    {\large \textbf{PARLIAMENTARY BRIEF}}
\end{center}

\vspace{0.3cm}

\begin{table}[h]
\centering
\begin{tabular}{@{}ll@{}}
\toprule
\textbf{For:} & Secretary, Department of Economic Affairs \\
\textbf{Ministry:} & Ministry of Finance \\
\textbf{Date:} & 27.10.2025 \\
\textbf{Prepared by:} & Parliamentary Affairs Division \\
\textbf{Reference:} & Lok Sabha Question No. 457, Winter Session 2025 \\
\bottomrule
\end{tabular}
\end{table}

\vspace{0.5cm}

% Subject
\briefheader{Subject:} Parliamentary Question on Implementation of New Financial Reporting Guidelines

\vspace{0.5cm}

% Question Details
\sectionheader{PARLIAMENTARY QUESTION:}
\datapoint{``Will the Secretary, Department of Economic Affairs please state the details of the new Financial Reporting Guidelines announced by the Ministry, including the implementation timeline, financial implications, and benefits for transparency in government financial management?''}

\vspace{0.5cm}

\sectionheader{1. BACKGROUND}

The Department of Economic Affairs has formulated comprehensive new Financial Reporting Guidelines (FRG) to modernize and standardize financial reporting across all Government of India Ministries and Departments. These guidelines have been developed in consultation with:

\begin{itemize}[leftmargin=*, itemsep=2pt]
    \item Comptroller and Auditor General of India
    \item Department of Expenditure
    \item Institute of Chartered Accountants of India
    \item International Monetary Fund (Technical Assistance)
\end{itemize}

The need for these reforms arises from:
\begin{itemize}[leftmargin=*, itemsep=2pt]
    \item Gap analysis of current reporting framework (last revised in 2010)
    \item International best practices in public financial management
    \item Recommendations of the Fifteenth Finance Commission
    \item Digital India initiative requirements
\end{itemize}

\vspace{0.5cm}

\sectionheader{2. CURRENT POSITION}

\subsection*{2.1 Guidelines Status}
\begin{itemize}[leftmargin=*, itemsep=2pt]
    \item Guidelines formally approved by Finance Minister on 20.10.2025
    \item Office Memorandum DEA/2025/OM/042 issued on 27.10.2025
    \item Implementation task force constituted under Joint Secretary (Budget)
\end{itemize}

\subsection*{2.2 Implementation Timeline}
\begin{table}[h]
\centering
\begin{tabular}{@{}ll@{}}
\toprule
\textbf{Phase} & \textbf{Key Milestones} \\
\midrule
Phase I (Jan-Mar 2026) & System preparation, training initiation \\
Phase II (Apr-Sep 2026) & Partial implementation, pilot testing \\
Phase III (Oct 2026 onwards) & Full implementation across all ministries \\
\bottomrule
\end{tabular}
\end{table}

\subsection*{2.3 Financial Implications}
\begin{itemize}[leftmargin=*, itemsep=2pt]
    \item Initial investment: \textbf{Rs. 245 crore} for system upgradation
    \item Annual recurring costs: \textbf{Rs. 78 crore} for maintenance and training
    \item Funding approved under Capital Account for FY 2025-26
    \item Expected efficiency savings: \textbf{Rs. 150 crore annually} by 2028
\end{itemize}

\vspace{0.5cm}

\sectionheader{3. KEY FACTS AND FIGURES}

\subsection*{3.1 Coverage and Scope}
\begin{table}[h]
\centering
\begin{tabular}{@{}lr@{}}
\toprule
\textbf{Parameter} & \textbf{Number} \\
\midrule
Total Ministries/Departments covered & 58 \\
Central Autonomous Bodies & 247 \\
Public Sector Enterprises (under DEA) & 42 \\
Annual transactions processed & 18.5 million \\
\bottomrule
\end{tabular}
\end{table}

\subsection*{3.2 International Comparisons}
\begin{itemize}[leftmargin=*, itemsep=2pt]
    \item \textbf{USA:} Federal Accounting Standards Advisory Board (FASAB) since 1990
    \item \textbf{UK:} Government Financial Reporting Manual (FReM) since 1998
    \item \textbf{Australia:} Australian Accounting Standards Board (AASB) since 1997
    \item \textbf{India:} Moving from cash-based to accrual-based accounting (first major reform)
\end{itemize}

\vspace{0.5cm}

\sectionheader{4. PROPOSED RESPONSE}

\subsection*{4.1 Main Response}
``The Government has formulated comprehensive new Financial Reporting Guidelines to enhance transparency and accountability in financial management. These guidelines will be implemented in a phased manner starting January 2026, with full operationalization by October 2026. The initiative represents the most significant reform in public financial management since 2010 and aligns India with international best practices.''

\subsection*{4.2 Key Benefits}
\begin{enumerate}[leftmargin=*, itemsep=2pt]
    \item \textbf{Enhanced Transparency:} Real-time financial position available to Parliament
    \item \textbf{Improved Accountability:} Clear tracking of financial commitments and liabilities
    \item \textbf{Better Decision Making:} Data-driven policy formulation based on accurate financial data
    \item \textbf{International Standards:} Compliance with International Public Sector Accounting Standards
    \item \textbf{Digital Integration:} Seamless integration with existing PFMS and IFMS systems
\end{enumerate}

\vspace{0.5cm}

\sectionheader{5. SUPPLEMENTARY INFORMATION}

\subsection*{5.1 For Potential Supplementary Questions}

\subsubsection*{Q1: What are the specific challenges in implementation?}
\textbf{A1:} The main challenges include:
\begin{itemize}[leftmargin=*, itemsep=2pt]
    \item Capacity building for over 15,000 finance officers across ministries
    \item Integration with legacy IT systems in certain departments
    \item Change management from cash-based to accrual-based accounting mindset
    \item Ensuring data consistency across diverse reporting entities
\end{itemize}

\subsubsection*{Q2: How will this affect Parliament's oversight functions?}
\textbf{A2:} Benefits to Parliamentary oversight:
\begin{itemize}[leftmargin=*, itemsep=2pt]
    \item Real-time access to financial data through Parliamentary Portal
    \item More accurate estimation of financial implications of Bills
    \item Better understanding of contingent liabilities and commitments
    \item Comparative analysis of inter-departmental financial performance
\end{itemize}

\subsubsection*{Q3: What measures are being taken for data security?}
\textbf{A3:} Security measures include:
\begin{itemize}[leftmargin=*, itemsep=2pt]
    \item End-to-end encryption of financial data transmission
    \item Multi-level authentication for system access
    \item Regular security audits by CERT-In empaneled agencies
    \item Compliance with IT Act 2000 and data protection regulations
\end{itemize}

\vspace{0.5cm}

\sectionheader{6. RELEVANT REFERENCES}

\begin{enumerate}[leftmargin=*, itemsep=2pt]
    \item Office Memorandum No. DEA/2025/OM/042 dated 27.10.2025
    \item Fifteenth Finance Commission Report, Volume I, Chapter 8
    \item Comptroller and Auditor General's Comments on Public Financial Management (2024)
    \item Cabinet Secretariat Instructions on Financial Governance (2023)
    \item International Public Sector Accounting Standards Board Framework
    \item Previous Lok Sabha Debate on Financial Transparency (Lok Sabha Debates, Vol. 23, 2024)
\end{enumerate}

\vspace{0.5cm}

\sectionheader{7. CONTACT INFORMATION}

For any follow-up questions or additional information:
\begin{itemize}[leftmargin=*, itemsep=2pt]
    \item \textbf{Point of Contact:} Joint Secretary (Budget), DEA
    \item \textbf{Email:} parliamentary.dea@gov.in
    \item \textbf{Phone:} 011-23742301
    \item \textbf{Documentation:} Available at www.finmin.nic.in/dea/financial-guidelines
\end{itemize}

\vspace{1cm}

\begin{center}
    {\small \color{restrictedred} \textbf{NOTE: This brief is prepared for internal parliamentary use and contains sensitive information.}}
\end{center}

\vspace{0.3cm}

\begin{flushright}
    \textbf{(Prepared by)}
    Parliamentary Affairs Division
    Department of Economic Affairs
    \textit{Date: 27.10.2025}
\end{flushright}

\end{document}